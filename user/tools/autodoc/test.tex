\documentclass[a4paper, 11pt]{article}
\usepackage[utf8]{inputenc}
\usepackage[english]{babel}
\usepackage[labelformat=empty]{caption}
\usepackage{geometry}
\usepackage{graphicx}
\usepackage{float}
\usepackage{fancyhdr}
\usepackage{lastpage}
\usepackage{titlesec}
\usepackage{hyperref}
\usepackage{minted}

\hypersetup{
 colorlinks=true, urlcolor=black, linkcolor=black,
 breaklinks=true, %permet le retour à la ligne dans les liens trop longs
 bookmarksopen=false, %si les signets Acrobat sont créés,
 % les afficher complètement.
 pdftitle={autodoc}, %informations apparaissant dans
 pdfauthor={Autodoc generated document}, %les informations du document
 pdfsubject={Documentation} %sous Acrobat.
}

\titlespacing\section{0mm}{8mm}{0mm}\titlespacing\subsection{4mm}{-8mm}{0mm}\titlespacing\subsubsection{8mm}{-12mm}{0mm}\geometry{hmargin=15mm,vmargin=25mm}
\setlength{\topmargin}{-30pt}
\setlength{\parindent}{0em}
\setlength{\parskip}{5mm}
\pagestyle{fancy}
\renewcommand{\headrulewidth}{1pt}
\lhead{autodoc}
\chead{September 2018} 
\rhead{\includegraphics[height=2cm]{/home/aurelien/Pictures/logo.jpg}}
\renewcommand{\footrulewidth}{1pt}
\lfoot{\leftmark}
\cfoot{}
\rfoot{\thepage\ / \pageref{LastPage}}

\newcommand{\nonumsection}[1]{
\newpage\paragraph{}\phantomsection
\addcontentsline{toc}{section}{#1}
\markboth{\uppercase{#1}}{}
\section*{#1}}

\newcommand{\nonumsubsection}[1]{
\paragraph{}\phantomsubsection
\addcontentsline{toc}{subsection}{#1}
\subsection*{#1}}

\newcommand{\nonumsubsubsection}[1]{
\paragraph{}\phantomsubsubsection
\addcontentsline{toc}{subsubsection}{#1}
\subsubsection*{#1}}

\newcommand{\file}[2]{
\nonumsubsection{#1}#2}

\begin{document}
\begin{titlepage}
\begin{center}
\includegraphics[height=8cm]{/home/aurelien/Pictures/logo.jpg}\\[1cm]
{\huge \bfseries autodoc}\\[8mm]
{\large September 2018}\\[16mm]
\begin{flushleft} \large An automatic documentation generator using emacs-lisp \end{flushleft}\\[-5mm]
\rule{\linewidth}{1pt}\\[5mm]
\begin{flushright} \Large
A. Buchet\\[3mm]
\large au.buchet@gmail.com\end{flushright}\\[2cm]
\end{center}
autodoc is an open source documentation generator using emacs-lisp to generate a .tex documentation file and can print it as a .pdf file using pdftex.
 autodoc is fully configurable and can be used with many languages and many templates.
\vfill
\end{titlepage}

\newpage\paragraph{}\tableofcontents\phantomsection
\markboth{CONTENTS}{}\newpage

\nonumsection{Prerequisites}
autodoc requires emacs and pdftex with some packages installed:\\inputenc\\babel\\caption\\geometry\\fancyhdr\\lastpage\\minted\\hyperref\\float

\nonumsection{Files}
\file{autodoc.el}{autodoc.el contains the whole autodoc script}

\nonumsubsubsection{printPDF}
printPDF takes a directory and a file in arguments and generates the pdf associated to file.tex in directory
\begin{minted}{emacs-lisp}
(defun printPDF (directory file) ... )
=> command
\end{minted}

\nonumsubsubsection{printPDF}
printPDF takes a directory and a file in arguments and generates the pdf associated to file.tex in directory
\begin{minted}{emacs-lisp}
(defun printPDF (directory file) ... )
=> command
\end{minted}

\newpage

\file{autodoc.el}{autodoc.el contains the whole autodoc script}

\nonumsubsubsection{printPDF}
printPDF takes a directory and a file in arguments and generates the pdf associated to file.tex in directory
\begin{minted}{emacs-lisp}
(defun printPDF (directory file) ... )
=> command
\end{minted}

\nonumsubsubsection{printPDF}
printPDF takes a directory and a file in arguments and generates the pdf associated to file.tex in directory
\begin{minted}{emacs-lisp}
(defun printPDF (directory file) ... )
=> command
\end{minted}

\newpage


\end{document}
