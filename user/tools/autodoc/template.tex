\documentclass[a4paper, 11pt]{article}

%% =================================================================================
%% Loading Packages
%% =================================================================================

\usepackage[utf8]{inputenc}
\usepackage[english]{babel}
\usepackage[labelformat=empty]{caption}
\usepackage{geometry}
\usepackage{graphicx}
\usepackage{float}
\usepackage{fancyhdr}
\usepackage{lastpage}
\usepackage{titlesec}
\usepackage{hyperref}
\usepackage{minted}

%% =================================================================================
%% Creating new commands
%% =================================================================================

\newcommand{\project}{@project}
\renewcommand{\date}{@date}
\newcommand{\logo}{@logo}
\addto\captionsenglish{\renewcommand\listfigurename{List of functions}}

\newcommand{\nonumsection}[1]{
\newpage\paragraph{}\phantomsection
\addcontentsline{toc}{section}{#1}
\markboth{\uppercase{#1}}{}
\section*{#1}}

\newcommand{\nonumsubsection}[1]{
\paragraph{}\phantomsubsection
\addcontentsline{toc}{subsection}{#1}
\subsection*{#1}}

\newcommand{\nonumsubsubsection}[1]{
\paragraph{}\phantomsubsubsection
\addcontentsline{toc}{subsubsection}{#1}
\subsubsection*{#1}}

%% =================================================================================
%% Document Settings
%% =================================================================================

\hypersetup{
 colorlinks=true, urlcolor=black, linkcolor=black,
 breaklinks=true, %permet le retour à la ligne dans les liens trop longs
 bookmarksopen=false, %si les signets Acrobat sont créés,
 % les afficher complètement.
 pdftitle={\project}, %informations apparaissant dans
 pdfauthor={Autodoc generated document}, %les informations du document
 pdfsubject={Documentation} %sous Acrobat.
}
   
\geometry{hmargin=1.5cm,vmargin=3cm}
\setlength{\topmargin}{-30pt}
\setlength{\parindent}{0em}
\setlength{\parskip}{1em}
\pagestyle{fancy}

%% =================================================================================
%% Headers and Footers
%% =================================================================================

\renewcommand{\headrulewidth}{1pt}
\lhead{\project}
\chead{\date} 
\rhead{\logo}
\renewcommand{\footrulewidth}{1pt}
\lfoot{\leftmark}
\cfoot{}
\rfoot{\thepage\ / \pageref{LastPage}}

%% =================================================================================
%% Contents
%% =================================================================================

\begin{document}

\begin{titlepage}
  \begin{center}
    \logo\\[1cm]
    {\Large \bfseries \project}\\[5mm]
    {\large \date}\\[5mm]
    \begin{flushleft} \large @summary \end{flushright}\\[5mm]
    \rule{\linewidth}{1pt}\\[5mm]
    {\large @authors}\\[5mm]
    {\large @contact}\\[2cm]
  \end{center}
  @abstract
  \vfill
\end{titlepage}

\newpage\tableofcontents
\newpage\nonumsection{Readme}
\newpage\nonumsection{Prerequisites}
\newpage\nonumsection{Installation}

\phantomsection
\addcontentsline{toc}{section}{\listfigurename}
\newpage\listoffigures\newpage

\begin{minted}{python}
import numpy as np
 
def incmatrix(genl1,genl2):
    m = len(genl1)
    n = len(genl2)
    M = None #to become the incidence matrix
    VT = np.zeros((n*m,1), int)  #dummy variable
 
    #compute the bitwise xor matrix
    M1 = bitxormatrix(genl1)
    M2 = np.triu(bitxormatrix(genl2),1) 
 
    for i in range(m-1):
        for j in range(i+1, m):
            [r,c] = np.where(M2 == M1[i,j])
            for k in range(len(r)):
                VT[(i)*n + r[k]] = 1;
                VT[(i)*n + c[k]] = 1;
                VT[(j)*n + r[k]] = 1;
                VT[(j)*n + c[k]] = 1;
 
                if M is None:
                    M = np.copy(VT)
                else:
                    M = np.concatenate((M, VT), 1)
 
                VT = np.zeros((n*m,1), int)
 
    return M
\end{minted}

\newpage


\begin{figure}\caption[mafonction]{}
  
\begin{minted}{emacs-lisp}
(defun mafonction nil)
=> nil
"Function to print YOLO"
\end{minted}
  
\end{figure}


\mint{emacs-lisp}|(defun mafonction (a b c yolo test abcdefghijklmnopqrst azertyuiop poiuytreza &rest yololo)) => output|

\newpage

test

\end{document}

